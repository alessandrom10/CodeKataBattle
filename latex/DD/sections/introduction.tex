\documentclass[../DD.tex]{subfiles}
\graphicspath{{\subfix{../assets}}}

\begin{document}
    \subsection{Purpose}\label{subsec:purpose}
    The purpose of this document is to provide a comprehensive technical overview of the system outlined in the RASD document. 
    In this context, we will evaluate the hardware and software architectures, emphasizing the interactions among the system's constituent components. 
    Furthermore, we will explore the details of the implementation, testing, and integration procedures. 
    While the document primarily employs technical language tailored for programmers, it extends an invitation to stakeholders for perusal, as it can offer valuable insights into the developmental aspects.

    \subsection{Scope}\label{subsec:scope}
    This design document outlines the system's behavior, encompassing both general and critical scenarios. 
    It delineates the system architecture by examining the logical assignment of components and their interactions.

    \subsection{Definitions, Acronyms, Abbreviations}\label{subsec:definitions_acronyms_abbreviations}
    \textbf{Acronyms}
    \begin{itemize}
        \item \textbf{RASD}: Requirements Analysis and Specification Document
        \item \textbf{DD}: Design Document
        \item \textbf{API}: Application Programming Interface
        \item \textbf{DBMS}: Database Management System
        \item \textbf{UML}: Unified Modeling Language
        \item \textbf{UI}: User Interface
        \item \textbf{GUI}: Graphical User Interface
        \item \textbf{HTTPS}: Hypertext Transfer Protocol Secure
        \item \textbf{HTML}: Hypertext Markup Language
        \item \textbf{CSS}: Cascading Style Sheets
        \item \textbf{JS}: JavaScript
        \item \textbf{MVC}: Model-View-Controller
    \end{itemize}

    \subsection{Revision History}\label{subsec:revision_history}
    \begin{itemize}
        \item Version 0.1: Setup
        \begin{itemize}
            \item[--] Created first layout 
        \end{itemize}
        \item Version 1.0: First release
        \begin{itemize}
            \item[--] Added section 1
            \item[--] Added section 2
            \item[--] Added section 3
            \item[--] Added section 4
            \item[--] Added section 5 
        \end{itemize}
    \end{itemize}

    \subsection{Reference Documents}\label{subsec:reference_documents}
    \begin{itemize}
        \item Specification document: "Assignment RDD AY 2023-2024"
        \item Requirements Analysis and Specification Document: "RASD"
        \item UML documentation: \url{https://www.uml-diagrams.org/}
    \end{itemize}

    \subsection{Document Structure}\label{subsec:document_structure}
    \begin{itemize}
        \item {\textbf{Section 1} offers a concise overview of the design document, outlining its purpose and scope, encompassing all the definitions, acronyms, and abbreviations employed.}
        \item {\textbf{Section 2} extensively explores the system architecture, offering an intricate depiction of the components, interfaces, and all the technical decisions undertaken for application development. 
        It encompasses detailed sequence, component, and ArchiMate\footnote{ArchiMate is an open and independent enterprise architecture modeling language to support the description, analysis and visualization of architecture within and across business domains in an unambiguous way\cite{ArchiMate}.} diagrams providing an in-depth understanding of the system.}
        \item {\textbf{Section 3} provides a comprehensive portrayal of the UI, including all client-side mockups alongside with graphs that explains the proper execution flow.}
        \item {\textbf{Section 4} maps the goals and requirements outlined in the RASD with the functionalities presented here\footnote{Here refers to the totality of this design document.}.}
        \item {\textbf{Section 5} presents the implementation, testing, and integration phases of the system components, which will be undertaken during the technical development of the application.}
    \end{itemize}

\end{document}
